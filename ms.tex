\documentclass[a4paper, 12pt]{article}
\usepackage{amsmath}
\usepackage{amssymb}
\usepackage{dsfont}
\usepackage[left=2cm, right=2cm, bottom=3cm, top=2cm]{geometry}
\usepackage{graphicx}
\usepackage{hyperref}
\usepackage[utf8]{inputenc}
\usepackage{microtype}
\usepackage{natbib}

\title{lbrycred}
\author{Brendon J. Brewer}
\date{}

\begin{document}
\maketitle

%\abstract{\noindent Abstract}

% Need this after the abstract
\setlength{\parindent}{0pt}
\setlength{\parskip}{8pt}

Let $\boldsymbol{y} = \{y_1, y_2, ..., y_n\}$ be the vector of credibility
scores (where $y_i \geq 0$) and $\boldsymbol{w} = \left\{w_{ij}\right\}$
a matrix of signed supports, where
$w_{ij}$ is the support from channel $i$ to channel $j$. Let the diagonal
elements $w_{ii}$ include the bid, self supports, and anonymous supports.
In practice the $\boldsymbol{w}$ will be sparse and concentrated along the
diagonal.

We need to compute the $\boldsymbol{y}$ values from the $\boldsymbol{w}$,
and Jeremy proposed an iterative procedure which reminds me of the
\href{https://en.wikipedia.org/wiki/Jacobi_method}{Jacobi Method} or
Gauss-Seidel. However,
I tried it out and it diverged --- presumably the matrix $\boldsymbol{w}$ had
eigenvalues above 1.

One can think of these iterative methods as optimisation techniques; basically
Gibbs sampling at zero temperature in order to optimise some objective
function. Since I'm better at optimisation than linear algebra, I will try to work in these terms. You can also think about these as numerical PDE solvers
like in undergrad COSC.

\section{Total Staked Amount}
Here I will try to rewrite the rule `credibility $\equiv$~total staked LBC'
as the solution to an optimisation problem, so I can see how to generalise it.
The total staked amount on channel $j$ is the row sum
\begin{align}
(\textnormal{LBC})_j = \sum_{i=1}^n w_{ij}.
\end{align}

Define the objective function $f(\boldsymbol{y})$ as
\begin{align}
f(\boldsymbol{y}) &= -\frac{1}{2}\sum_{j=1}^n y_j^2.
\end{align}
That's maximised by setting all elements of $\boldsymbol{y}$ to zero.
Now add another term:
\begin{align}
f(\boldsymbol{y}) &= -\frac{1}{2}\sum_{j=1}^n
                            \left(y_j - (\textnormal{LBC})_j\right)^2.
\end{align}
This just moves the peak so the maximum is trivially found at
$y_j = (\textnormal{LBC})_j$.


\bibliographystyle{plainnat}
\bibliography{references}

\end{document}

