\documentclass[a4paper, 12pt]{article}
\usepackage{amsmath}
\usepackage{amssymb}
\usepackage{dsfont}
\usepackage[left=2cm, right=2cm, bottom=3cm, top=2cm]{geometry}
\usepackage{graphicx}
\usepackage{hyperref}
\usepackage[utf8]{inputenc}
\usepackage{microtype}
\usepackage{natbib}
\newcommand{\lbc}{\ell}


\title{lbrycred}
\author{Brendon J. Brewer}
\date{}

\begin{document}
\maketitle

%\abstract{\noindent Abstract}

% Need this after the abstract
\setlength{\parindent}{0pt}
\setlength{\parskip}{8pt}

Let $\boldsymbol{y} = \{y_1, y_2, ..., y_n\}$ be the vector of credibility
scores (where $y_i \geq 0$) and $\boldsymbol{w} = \left\{w_{ij}\right\}$
a matrix of signed supports, where
$w_{ij}$ is the support from channel $i$ to channel $j$. Let the diagonal
elements $w_{ii}$ include the bid, self supports, and anonymous supports.
In practice the $\boldsymbol{w}$ will be sparse and concentrated along the
diagonal. The total staked amount on channel $j$ is the column sum
\begin{align}
\ell_j = \sum_{i=1}^n w_{ij}.\label{eqn:total_lbc}
\end{align}
It would be nice if $\boldsymbol{y}$ were in units of LBC and if
it was normalised to the same total as the staked LBC:
\begin{align}
\sum_{i=1}^n \ell_i = \sum_{i=1}^n y_i,
\end{align}
and what I propose here and demonstrate in {\tt simulate.py} satisfies this
criterion.

We need to compute the $\boldsymbol{y}$ values from the $\boldsymbol{w}$,
and Jeremy (or someone else at LBRY Inc) proposed an iterative procedure which
reminds me of the
\href{https://en.wikipedia.org/wiki/Jacobi_method}{Jacobi Method} or
(more accurately) Gauss-Seidel. However,
I tried it out and it diverged --- presumably the matrix $\boldsymbol{w}$ had
eigenvalues above 1.

{\bf Aside mostly for my own thinking}:
One can think of these iterative methods as optimisation techniques; basically
Gibbs sampling at zero temperature in order to optimise some objective
function. Since I'm better at optimisation than linear algebra, I will try to
work in these terms. You can also think about these as numerical PDE solvers
like in undergrad COSC.


\section{Taking Supporter Credibility Into Account}
Now let's try replacing $\ell_j$ with an alternative that accounts for the
credibilities of the supporting channels. We'll need to treat the diagonal
value $w_{jj}$ differently because self and anonymous supports don't get
modified due to credibility.

To do this, I will introduce another quantity for each channel,
$\boldsymbol{m} = \{m_1, m_2, ..., m_n\}$. The {\em m} stands for
{\em multiplier}, as the effect of these is going to be multiplication of LBC
(so if a channel has $m=2$, their staked LBC is twice as valuable as someone
with $m=1$). All $m$ values are greater than or equal to 1.
Before continuing, let's define a nonlinear softening function which takes
an LBC amount as input and returns a multiplier value. Similar to trending,
I will use a power, with a modification so that the initial value
is 1:
\begin{align}
\Phi(x) &= (x + 1)^\alpha.
\end{align}
The inverse of this is
\begin{align}
\Phi^{-1}(x) &= x^{\frac{1}{\alpha}} - 1,
\end{align}
and in the demo I have set $\alpha = 1/3$.

If the $y$ values are correct, the $m$-values are:
\begin{align}
m_j &= \Phi(y_j). \label{eqn:grade}
\end{align}

The iterative procedure to compute the $m$s and thus the $y$s is as follows,
starting from an initial estimate of the $y$s which should probably be set
equal to the $\ell$s (the total staked LBC).
\begin{enumerate}
\item Compute the $m$s from the $y$s using Equation~\ref{eqn:grade}.
\item Update the $y$s using $y_j = w_{jj} + \sum_{i \neq j} m_i w_{ij}$
($\equiv$~Eqn~\ref{eqn:total_lbc} if $m$s are 1)
\item Re-normalise the $y$s by dividing by their sum and multiplying by total
staked LBC
\end{enumerate}
This process is repeated until the change in the $y$s is neligible. Typically
not many iterations are needed.

\bibliographystyle{plainnat}
\bibliography{references}

\end{document}

